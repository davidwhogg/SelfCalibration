\documentclass[12pt,preprint,dvips]{aastex}

\newcommand{\foreign}[1]{\textsl{#1}}
\newcommand{\etal}{\foreign{et~al.}}
\newcommand{\project}[1]{\textsl{#1}}
\newcommand{\SDSS}{\project{SDSS}}

\begin{document}

\section*{Can we infer the pixel-level flat-field without an internal calibration source?}

\noindent
David W. Hogg (NYU)
\\
\textsl{Max-Planck-Institut f\"ur Astronomie, Heidelberg, Germany}

We have shown with large optical imaging surveys that the redundant
information coming from the multiple observations of stars is
sufficient to infer the large-scale sensitivity (effective area or
flat-field or vignetting) of an astronomical imaging device
(\citealt{ubercal}, \citealt{holmes}).  In these self-calibration
experiments, the sensitivity function has had only hundreds or
thousands of parameters.  We have not attempted to obtain millions of
pixel-level parameters from the data and we have not considered
arbitrarily non-smooth two-dimensional sensitivity functions.

With observations of billions of sources on the sky, however, it might
be possible to infer millions of sensitivity parameters, even as many
or more sensitivity parameters than there are pixels in the imaging
device.  After all, we know a lot about the sky, many sources are
observed multiple times, and every source touches multiple pixels even
in a single visit.

In detail, the sensitivity of a pixel can be a function of spectral
energy distribution (different pixels will be differently sensitive to
sources with different colors or spectra) and a function of
intra-pixel illumination pattern (the pixel might have a more
sensitive region and some less sensitive regions).  Also, diffuse
illumination, such as uniform sky background, might illuminate the
pixel from a different mixture of angles than compact illumination,
such as from a star, so there might be, effectively, different pixel
sensitivities for the sky and for sources.  Indeed, reflections in the
\SDSS\ camera led to problems of this kind, which invalidated the
sky-based flat-fields (\citealt{ubercal}).  For all these reasons we
might actually have many sensitivity parameters per pixel; this makes
our task harder quantitatvely, but perhaps not qualitatively.

\begin{thebibliography}{70}
\bibitem[Padmanabhan \etal(2008)]{ubercal}
Padmanabhan,~N., Schlegel,~D.~J., Finkbeiner,~D.~P., \etal, 2008, \apj, 674, 1217
\bibitem[Holmes \etal(2012)]{holmes}
Holmes,~R., Hogg,~D.~W., \& Rix,~H.-W., 2012, arXiv:1203.6255 
\end{thebibliography}

\end{document}
