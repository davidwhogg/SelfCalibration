\documentclass[12pt,preprint,dvips]{aastex}

\newcommand{\foreign}[1]{\textsl{#1}}
\newcommand{\etal}{\foreign{et~al.}}

\begin{document}

\paragraph{Can we infer the pixel-level flat-field without an internal calibration source?}

We have shown with large surveys that the redundant information coming
from the multiple observations of stars is sufficient to infer the
large-scale sensitivity (effective area or flat-field or vignetting)
of an imaging device (\citealt{ubercal}, \citealt{holmes}).  In these
experiments, the photometry has been thought of as being well sampled,
and the flat-field function has had only thousands of parameters. We
have not attempted to obtain millions of pixel-level parameters from
the data and we have not considered arbitrarily non-smooth functions
of two-dimensional position.

With observations of billions of sources on the sky, however, it might
be possible to infer millions of sensitivity parameters, even as many
or more sensitivity parameters than there are pixels in the imaging
device.  After all, we know a lot about the sky, many sources are
observed multiple times, and every source touches multiple pixels even
in a single visit.

\begin{thebibliography}{70}
\bibitem[Padmanabhan \etal(2008)]{ubercal}
Padmanabhan,~N., Schlegel,~D.~J., Finkbeiner,~D.~P., \etal, 2008, \apj, 674, 1217
\bibitem[Holmes \etal(2012)]{2012arXiv1203.6255H}
Holmes,~R., Hogg,~D.~W., \& Rix,~H.-W., 2012, arXiv:1203.6255 
\end{thebibliography}

\end{document}
