\documentclass[12pt,preprint,dvips]{aastex}

\newcommand{\foreign}[1]{\textsl{#1}}
\newcommand{\etal}{\foreign{et~al.}}
\newcommand{\project}[1]{\textsl{#1}}
\newcommand{\SDSS}{\project{SDSS}}

\newcommand{\expectation}[1]{\tilde{#1}}
\newcommand{\sky}{^{(\mathrm{sky})}}
\newcommand{\source}{^{(\mathrm{src})}}

\begin{document}

\section*{Can we infer the pixel-level flat-field without an internal calibration source?}

\noindent
David W. Hogg (NYU)
\\{\footnotesize
\textsl{Center for Cosmology and Particle Physics, Department of Physics, New York University, New York, USA}
\\
\textsl{Max-Planck-Institut f\"ur Astronomie, Heidelberg, Germany}}

\textsl{This document is a draft dated 2012-08-20.  It is based in
  part on conversations with Rob Fergus (NYU), Rory Holmes (MPIA), and
  Bernhard Sch\"olkopf (MPI-IS).}

We have shown with large optical imaging surveys that the redundant
information coming from the multiple observations of stars is
sufficient to infer the large-scale sensitivity (effective area or
flat-field or vignetting) of an astronomical imaging device
(\citealt{ubercal}, \citealt{schlafly}, \citealt{holmes}).  In these
self-calibration experiments, the sensitivity function has had only
hundreds or thousands of parameters.  We have not attempted to obtain
millions of pixel-level parameters from the data and we have not
considered arbitrarily non-smooth two-dimensional sensitivity
functions.

With observations of billions of sources on the sky, however, it might
be possible to infer millions of sensitivity parameters, even as many
or more sensitivity parameters than there are pixels in the imaging
device.  After all, we know a lot about the sky, many sources are
observed multiple times, and every source touches multiple pixels even
in a single visit.

In detail, the sensitivity of a pixel can be a function of spectral
energy distribution (different pixels will be differently sensitive to
sources with different colors or spectra) and a function of
intra-pixel illumination pattern (the pixel might have a more
sensitive region and some less sensitive regions).  Also, diffuse
illumination, such as uniform sky background, might illuminate the
pixel from a different mixture of angles than compact illumination,
such as from a star, so there might be, effectively, different pixel
sensitivities for the sky and for sources.  Indeed, reflections in the
\SDSS\ camera led to problems of this kind, which invalidated the
sky-based flat-fields (\citealt{ubercal}).  For all these reasons we
might actually have many sensitivity parameters per pixel; this makes
our task harder quantitatvely, but perhaps not qualitatively.

\paragraph{varying levels of dumbness:}
There are various thought experiments or considerations that suggest
that a pixel-level self-calibration is possible.  One observation is
that if a pixel is unusually low in sensitivity, it's mean output
signal will be low on average for the whole data set.  This statement
could be made more precise for the sky flat by considering only
observations in which the pixel is not near any detectable source.  It
could be made even more precise if each observation is divided by the
mean pixel value, to account for variations in sky brightness.  These
kinds of operations or considerations could lead to a very good sky
flat for any device, and is not unlike what is usually done.

As we said, however, each pixel might have a different response to
compact sources than it does to the diffuse sky, so slightly less-dumb
methods might be called for.  For example, consider all overlapping
$3\times 3$-pixel patches in a full data set.  Most of these patches
would be far from any detectable source and could be labeled ``sky''.
A fraction would be near or on top of compact, detected sources and
could be labeled ``source''.  This labeling could be
non-comprehensive; there could be intermediate patches labeled
``neither''; this is just an engineering consideration.  Less dumb
still would be to make bigger patches (say $5\times 5$ or larger) or
to make use of the residuals away from a physically motivated model of
the observed images.

\paragraph{pixel signal models:}
Sticking with the intermediate level of dumbness for now: In a
$3\times 3$ sky patch (a patch far from any detectable sources), the
expected signal $\expectation{D}_{ij}$---the expected number of
electrons or data numbers read out---in pixel $i$ of exposure $j$
might be a combination of zero-level, dark-current, and sky-photon
signals
\begin{eqnarray}
\expectation{D}_{ij} = z_i + d_i\,t_j + f\sky_i\,S\sky_j\,t_j
\quad ,
\end{eqnarray}
where $i$ labels the device pixel ($i$ maps back to a pixel location
in the original imaging device), $j$ labels the exposure, each of
which might have a slightly different sky level, $z_i$ is a zero-level
or bias for pixel $i$, $d_i$ is a dark current rate or signal per time
in the absence of illumination, $t_j$ is the exposure time for
exposure $j$, $f\sky_i$ is the sensitivity of pixel $i$ to diffuse sky
illumination, and $S\sky_j$ is the intensity of the sky in exposure
$j$ (interpolated to pixel $i$ if there are gradients).

In a $3\times 3$ source patch (a patch containing some detectable
source or part thereof), the expectation is similar but there is now
source flux
\begin{eqnarray}
\expectation{D}_{ij} = z_i + d_i\,t_j + f\sky_i\,S\sky_j\,t_j
                     + f\source_i\,S\source_{ij}\,t_j
\quad ,
\end{eqnarray}
where now $f\source_i$ is the sensitivity of pixel $i$ to compact
source flux, and $S\source_{ij}$ is the source flux in exposure $j$
interpolated to or evaluated at the location of pixel $i$.  This
interpolation to pixel $i$ is where the $3\times 3$ pixel patch comes
in: We are going to use the 8 pixels surrounding pixel $i$ to
interpolate the source flux probabilistically to pixel $i$.  This also
shows that using a bigger patch, say $5\times 5$, would be
substantially less dumb.  Even the sky patches (as noted momentarily
above) might require interpolation to pixel $i$, since there might be
sky gradients.

In the case that there are also \emph{dark}
observations---observations with the shutter closed---or \emph{zero}
observations---observations with zero exposure times---there can be
additional constraints on the $z_i$ and the $d_i$ beyond just the
science-data constraints.  These considerations also suggest that a
survey that varies its exposure time will thereby obtain much more
information about the calibration information (and also vet the
shutter!).

There might be an additional dependence of the source sensitivity
$f\source_i$ on color or spectral energy distribution.  There might
also be variations of all the calibration parameters on time, possibly
even with break-points or sharp transitions (think episodic radiation
damage and the like).  There will also be many, many pixels affected
by cosmic rays.  All of these issues make the real-world execution of
this program more complicated than the cartoon given here.  For now we
might try assuming that we can work with only patches that are
obviously \emph{not} affected by cosmic rays, and work only with
time-slices of the data within which no significant variations are
expected.

The likelihood of any model for the data will be related to the idea
that the observed or read-out data $D_{ij}$ differs from the model
expectation $\expectation{D}_{ij}$ by the contribution of some kind of
noise, which might be Poisson or Gaussian or some combination,
depending on the device and our precision needs.

\paragraph{blind deconvolution:}
This problem looks very much like a blind deconvolution problem,
because there is a true but unknown scene multiplied by some unknown
set of pixel responses before noisy detection.  If the astronomical
fields $(S\sky_j, S\source_{ij})$ are known, the pixel response
parameters $(z_i, d_i, f\sky_i, f\source_i)$ are related to the signal
by a pure linear operation and \foreign{vice versa}.  If the noise in
the data is Gaussian (or can be approximated as such), the inference
can proceed by iterated weighted linear least squares.  If the imager
is well-behaved, $S\sky_j$ will be a very smooth function of pixel
position $i$ or no function at all.  If the imaging is well-sampled,
$S\source_{ij}$ will be expressable as a linear combination of
band-limited basis functions, like sincs or Gaussians.

If we aren't satisfied with point estimates---and we shouldn't
be---then we can propagate the uncertainties in both the calibration
quantities and the astronomical scene as a set of samples or explicit
analytic descriptions of the posterior PDF.  The latter might be
possible in the case of pure Gaussian noise.

\paragraph{tests with real data}
We have identified the \project{HAWK-I} camera on \project{Paranal} as
a possible analog.  It appears to have flat-field morphology similar
to that expected for the \project{Euclid} infrared camera, and it is
the same detector technology.  The next steps are
\begin{itemize}
\item Find and download a range of science data taken in one chosen
  filter with a range of exposure times.
\item Choose a small patch of the detector where there is known to be
  good or interesting flat-field structure.
\item Make source and sky patches from those data, and do some
  analysis.
\end{itemize}

\begin{thebibliography}{70}
\bibitem[Holmes \etal(2012)]{holmes}
Holmes,~R., Hogg,~D.~W., \& Rix,~H.-W., 2012, arXiv:1203.6255 
\bibitem[Padmanabhan \etal(2008)]{ubercal}
Padmanabhan,~N., Schlegel,~D.~J., Finkbeiner,~D.~P., \etal, 2008, \apj, 674, 1217
\bibitem[Schlafly \etal(2012)]{schlafly}
Schlafly,~E.~F., Finkbeiner,~D.~P., Juric,~M., \etal, 2012, arXiv:1201.2208 
\end{thebibliography}

\end{document}
