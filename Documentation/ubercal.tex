\documentclass[12pt,a4paper,twoside]{article}
\usepackage[latin1]{inputenc}
\usepackage{amsmath}
\usepackage{amsfonts}
\usepackage{amssymb}
\usepackage{color}
\usepackage{listings}

\author{David Hogg, Rory Holmes}
\title{Ubercalibration Simulations \\(Euclid)}
\date{Jan 2011}

\makeatletter
\def\maketitle{%
  \null
  \thispagestyle{empty}%
  \vfill
  \begin{center}\leavevmode
    \normalfont
    {\LARGE \@title\par}%
    \vskip 1cm
    {\Large \@author\par}%
    \vskip 1cm
    {\Large \@date\par}%
  \end{center}%
  \vfill
  \null
  \cleardoublepage
  }
\makeatother

\begin{document}
\maketitle

\section{Introduction}
These simulations... 

\section{General}

\subsection{Parameter File}

Simulation parameters are stored in the parameter.py file.

\subsection{General Functions}
The functions.py file contains two routine for converting between magnitudes and fluxes. ..

\section{Sky Catalog}
The 

\section{Camera Model}

\subsection{Sky to Focal Plane Coordinate Transformations}

\subsection{Uncertainty Inverse-Variance}

\subsection{Catalog to Measured Flux Conversion}

\section{Ubercalibration}
The ubercalibration procedure works in two sub-steps: (i) the star fluxes are refined based on the latest flat-field estimate (\textit{s\_step}) and (ii) the flat-field parameters are refined based on the latest star flux estimates (\textit{q\_step}). We iterate these steps until the parameters converge. In our notation $j$ refers to the iteration step. 

Our flat-field is defined by a set of parameters $\vec{q}$. After the $j$th iteration, the flat-field parameters will be indexed as $\vec{q_j}$. The individual star measurements are defined as $c_i$ (``counts''), where each $i$ corresponds to a exposure number $n$, a specific star with ID number $k$ and a focal plane position $\vec{x}$.

XX Our model is

\begin{equation}
c_i = f(\vec{x_i} | \vec{q_j}) . s_{kj} + e_{ij}
\end{equation}

where $s_k$ is the true flux from star $k$ and our error is drawn from the Normal Distribution $e_{ij} = N(e|0,{\sigma_i}^2)$.

\subsection{Star Flux Refinement: \textbf{\textit{s\_step}}}
The flux estimates, $s_k$, for each star can be considered individually. At step $j$ we can refine a star's flux estimate (to $s_{kj+1}$) by minimizing the error function with the latest flat-field parameters:

\begin{equation}
\chi^2_{k} = \sum_{i \in \mathcal{O}(k)} \frac{(c_i-f_{ij}s_{kj})^2}{{\sigma_i}^2}
\end{equation}

\begin{equation}
\frac{d\chi^2_{k}}{d s_{kj}} = \sum_{i \in \mathcal{O}(k)} \frac{-2 f_{ij} (c_i-f_{ij}s_{kj+1})}{{\sigma_i}^2} = 0
\end{equation}

\begin{equation}
\Rightarrow \sum_{i \in \mathcal{O}(k)} \frac{f_{ij} c_i}{{\sigma_i}^2}= \sum_{i \in \mathcal{O}(k)} \frac{f_{ij}^2 s_{kj+1}}{{\sigma_i}^2}
\end{equation}

\begin{equation}
\Rightarrow s_{kj+1} = {\sum_{i \in \mathcal{O}(k)} \frac{f_{ij} c_i}{{\sigma_i}^2}}/{\sum_{i \in \mathcal{O}(k)} \frac{f_{ij}^2}{{\sigma_i}^2}}
\end{equation}

\subsection{Flat-Field Refinement: \textbf{\textit{p\_step}}}
The flat-field parameters can now be refined with the latest star flux estimates. Minimise error function with respect to the...

Consider all measurements

\begin{equation}
\chi^2 = \sum_{all~i} \frac{(c_i-f_{ij}s_{kj})^2}{{\sigma_i}^2}
\end{equation}

\begin{equation}
f_{ij} = \sum_{l = 1}^L q_{lj} g_i(\vec{x_i})
\end{equation}

\begin{equation}
\chi^2 = \sum_{all~i} \frac{(c_i- s_{kj} \sum_{l = 1}^L q_{lj} g_i(\vec{x_i}))^2}{{\sigma_i}^2}
\end{equation}

\begin{equation}
\frac{d\chi^2}{dq_{lj}} = \sum_{all~i} \frac{-2 g_l(\vec{x_i}) s_{kj+1} (c_i- s_{kj} \sum_{l = 1}^L q_{lj} g_i(\vec{x_i}))}{{\sigma_i}^2} = 0
\end{equation}

\begin{equation}
\sum_{all~i} \frac{g_l(\vec{x_i}) s_{kj+1} c_i}{{\sigma_i}^2} = \sum_{all~i} \frac{g_l(\vec{x_i}) s_{kj+1}^2 \sum_{l = 1}^L q_{lj+1} g_l(\vec{x_i})}{{\sigma_i}^2}
\end{equation}

\end{document}
