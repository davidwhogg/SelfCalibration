\documentclass[manuscript]{aastex}
\newcommand{\vdag}{(v)^\dagger}

\usepackage{mathtools}
\usepackage{upgreek}

\newcommand{\true}{\text{true}}
\newcommand{\fit}{\text{fit}}
\newcommand{\basis}{\text{best}}
\newcommand{\dd}{\text{d}}

\shorttitle{Optimizing Large Scale Imaging Surveys for Relative Photometric Calibration}
\shortauthors{Holmes et al.}

%% This is the end of the preamble.  Indicate the beginning of the
%% paper itself with \begin{document}.

\begin{document}

%% LaTeX will automatically break titles if they run longer than
%% one line. However, you may use \\ to force a line break if
%% you desire.

\title{Optimizing Large Scale Imaging Surveys for Relative Photometric Calibration, \\
    the Euclid Dark Energy Mission as an Example}

%% Use \author, \affil, and the \and command to format
%% author and affiliation information.
%% Note that \email has replaced the old \authoremail command
%% from AASTeX v4.0. You can use \email to mark an email address
%% anywhere in the paper, not just in the front matter.
%% As in the title, use \\ to force line breaks.

\author{R. Holmes\altaffilmark{1} and H.-W. Rix\altaffilmark{1}}
\affil{Max-Planck-Institut f\"ur Astronomie, K\"onigstuhl 17, Heidelberg, 69117, Germany.}

\and

\author{D. W. Hogg\altaffilmark{2}}
\affil{2 Center for Cosmology and Particle Physics, Department of Physics, New York University, 4 Washington Place, New York, NY 10003, USA.}


\begin{abstract}
In this paper we show that, with due care given to the observing strategy, it will be possible to accurately constrain the relative photometric calibration of instruments used in large imaging surveys with their science data alone. We create end-to-end simulations of an imaging survey, which produces simulated datasets from mock observations of a synthetic sky according to a defined survey strategy. These catalog based simulations include realistic measurement uncertainties and a complex, position dependent instrument response. We then use a self-calibration technique to recover the relative instrument response by fitting an model that best explains the survey dataset, based on the multiple observations of (non-varying) sources at different focal plane positions. By considering four simple survey strategies we find that, with a correct redundancy built into the survey strategy, it is possible to accurately constrain the relative photometric response of an imaging instrument, and therefore the relative calibration of the resulting dataset. The majority of the remaining post self-calibration errors are due to the limitations in the basis used to model the relative instrument response. We find that returning the same sources to very different focal plane positions is the key property of a survey strategy that is required for an accurate calibration. From the results of this work, we are able to highlight an important point for those considering the design of large scale imaging surveys: depart from a regular tiling of the sky and return the same sources to very different focal plane positions.
\end{abstract}

\keywords{Relative Photometric Calibration: Imaging Survey: Euclid}

%% From the front matter, we move on to the body of the paper.
%% In the first two sections, notice the use of the natbib \citep
%% and \citet commands to identify citations.  The citations are
%% tied to the reference list via symbolic KEYs. The KEY corresponds
%% to the KEY in the \bibitem in the reference list below. We have
%% chosen the first three characters of the first author's name plus
%% the last two numeral of the year of publication as our KEY for
%% each reference.


%% Authors who wish to have the most important objects in their paper
%% linked in the electronic edition to a data center may do so by tagging
%% their objects with \objectname{} or \object{}.  Each macro takes the
%% object name as its required argument. The optional, square-bracket 
%% argument should be used in cases where the data center identification
%% differs from what is to be printed in the paper.  The text appearing 
%% in curly braces is what will appear in print in the published paper. 
%% If the object name is recognized by the data centers, it will be linked
%% in the electronic edition to the object data available at the data centers  
%%
%% Note that for sources with brackets in their names, e.g. [WEG2004] 14h-090,
%% the brackets must be escaped with backslashes when used in the first
%% square-bracket argument, for instance, \object[\[WEG2004\] 14h-090]{90}).
%%  Otherwise, LaTeX will issue an error. 

\section{Introduction}
Astronomers tend to think in terms of taking science data and calibration data separately. The former is used for science and the latter is used to constrain instrument parameters like the flat-field, the dark currents and so-on. But typically far more photons are collected during science exposures; are these not incredibly constraining on the calibration themselves? Indeed, in the retrospective photometric calibration of the Sloan Digital Sky Survey data (SDSS) much more calibration information was obtained from the science data than the calibration data \citep{pad08}. But, of course, the SDSS imaging strategy had to be adjusted to make this calibration work: good redundancy is required in the data stream, and a redundancy of a very specific kind.

In this paper, we argue that the next generation of large surveys should have their observation strategies optimized from the very start with this ``self-calibration'' in mind. We focus here on the relative photometric calibration of a typical imaging instrument only, although we also suspect that similar techniques could be used to constrain many other calibration parameters. This technique utilizes the multiple measurements of (non-varying) sources at different focal plane positions and at different times within the survey to constrain the relative instrument response by requiring that they yield the same count values. Through end-to-end -- catalog level -- survey simulations\footnote{All of the code used in this work is publicly available at XX.}, we aim to identify the important properties of survey strategies that makes them advantageous for this kind of self-calibration. 

These simulations include a complex, position dependent instrument response; through mock observations of a synthetic sky, according to a defined survey strategy, we construct a realistic survey dataset. We then fit a (different) instrument response model that best describes this dataset. By comparing the \textit{fitted} instrument response to the \textit{true} instrument response we are able to assess the performance of the self-calibration procedure with different survey strategies. There is a degeneracy in this problem: the self-calibration procedure is only able to fit for a \textit{relative} instrument response. We have no way of know, for example, if all the sources are uniformly fainter, or if the instrument sensitivity is uniformly lower. 

In Section \ref{sec:survey} we introduce the simulation chain constructed to produce the realistic survey datasets. Section \ref{sec:self_cal} goes onto to detail the self-calibration procedure. In Section \ref{sec:simple} we focus on four simple survey strategies, which allow us to draw conclusions on the performance of the self-calibration procedure with different survey properties. We do not produced pixelated images; instead we concentrate on catalog level simulations with realistic measurement uncertainties. Complex effects are included in the simulations that are not  precisely modeled at the analysis stage, in order to simulate the effects of unknown systematic errors within the dataset.

\section{Survey Dataset Simulations}
\label{sec:survey}
We have constructed an end-to-end simulation chain that produces a realistic imaging dataset from a specified survey strategy. Simulation parameters are kept intentionally flexible, so that the sensitivity of the self-calibration procedure to different parameters can be investigated. The simulations are split into a number of steps. First, a synthetic sky is generated based on the specified simulation parameters (density of sources, magnitude ranges). With a given pointing, single mock observations are performed on this sky, again here with tunable simulation parameters, such as the instrument response model and the the noise model. The final step is the build up of the survey wide dataset from multiple single observations taken according to the prescribed survey strategy. In this section, we detail the assumptions and methods used in each of these steps. 

\subsection{The Synthetic Sky}
We generate a representative synthetic sky based on realistic object densities in the AB magnitude range $m_\text{min} = 17$ to $m_\text{max} = 22$~mag$_\text{AB}$, with these limits chosen to be consistent with the saturation and $10\upsigma{}$ limits of deep, space-based, near-infrared survey. Sources are generated with random coordinates (uniformly distributed within the sky region being investigated) and with random magnitudes $m$ distributed according to
\begin{eqnarray}
\log_{10} \frac{\dd N}{\dd m \, \dd \Omega} = a + b\,m + c\,m^2 \label{eqn:power_law} \quad , 
\end{eqnarray}
where $\dfrac{\dd N}{\dd m \, \dd  \Omega}$  is the density of sources $N$ per unit magnitude $m$ and per unit solid angle $\Omega$, and $a$, $b$ and $c$ are model parameters. Even though our simulations make no distinction between galaxies and stars, the values of the parameters are found from fitting the Y-band galaxy populations reported in Windhorst et al. (2011) \cite{source_densities} only. These parameters were $a = -13.05$, $b = 1.25$ and $c = -0.02$. We intentionally omit the stellar population, so that the conclusions drawn from these simulations are independent of the final position of the survey area on the sky. The source magnitudes $m$ are related to the source fluxes $s$ simply by: $m = 22.5 - 2.5\log_{10}(s)$, where the 22.5 puts the fluxes in units of nanomaggies (nmgy).

\subsection{A Single Exposure}
FoV

\subsection{The Complete Survey}


\section{The Self-Calibration Technique}
\label{sec:self_cal}

\section{The Simple Survey Strategies}
\label{sec:simple}

\section{Discussion}
In this paper, we have considered the optimization of survey strategies for calibrating the relative photometric response of an instrument, but we feel there are many more calibration parameters that can be constrained for this method. For example, the optical distortion of an instrument could also be constrained with such a method, although we concede that the properties of a survey strategy that makes them good for one calibration, may not be the same as that for another; ultimately a trade-off would need to be performed. 





%% If you wish to include an acknowledgments section in your paper,
%% separate it off from the body of the text using the \acknowledgments
%% command.

%% Included in this acknowledgments section are examples of the
%% AASTeX hypertext markup commands. Use \url without the optional [HREF]
%% argument when you want to print the url directly in the text. Otherwise,
%% use either \url or \anchor, with the HREF as the first argument and the
%% text to be printed in the second.

\acknowledgments
\section{Acknowledgments}
We are grateful to 

%% To help institutions obtain information on the effectiveness of their
%% telescopes, the AAS Journals has created a group of keywords for telescope
%% facilities. A common set of keywords will make these types of searches
%% significantly easier and more accurate. In addition, they will also be
%% useful in linking papers together which utilize the same telescopes
%% within the framework of the National Virtual Observatory.
%% See the AASTeX Web site at http://aastex.aas.org/
%% for information on obtaining the facility keywords.

%% After the acknowledgments section, use the following syntax and the
%% \facility{} macro to list the keywords of facilities used in the research
%% for the paper.  Each keyword will be checked against the master list during
%% copy editing.  Individual instruments or configurations can be provided 
%% in parentheses, after the keyword, but they will not be verified.

%% Appendix material should be preceded with a single \appendix command.
%% There should be a \section command for each appendix. Mark appendix
%% subsections with the same markup you use in the main body of the paper.

%% Each Appendix (indicated with \section) will be lettered A, B, C, etc.
%% The equation counter will reset when it encounters the \appendix
%% command and will number appendix equations (A1), (A2), etc.

\appendix

%% The reference list follows the main body and any appendices.
%% Use LaTeX's thebibliography environment to mark up your reference list.
%% Note \begin{thebibliography} is followed by an empty set of
%% curly braces.  If you forget this, LaTeX will generate the error
%% "Perhaps a missing \item?".
%%
%% thebibliography produces citations in the text using \bibitem-\cite
%% cross-referencing. Each reference is preceded by a
%% \bibitem command that defines in curly braces the KEY that corresponds
%% to the KEY in the \cite commands (see the first section above).
%% Make sure that you provide a unique KEY for every \bibitem or else the
%% paper will not LaTeX. The square brackets should contain
%% the citation text that LaTeX will insert in
%% place of the \cite commands.

%% We have used macros to produce journal name abbreviations.
%% AASTeX provides a number of these for the more frequently-cited journals.
%% See the Author Guide for a list of them.

%% Note that the style of the \bibitem labels (in []) is slightly
%% different from previous examples.  The natbib system solves a host
%% of citation expression problems, but it is necessary to clearly
%% delimit the year from the author name used in the citation.
%% See the natbib documentation for more details and options.

\begin{thebibliography}{}
\bibitem[Padmanabhan et al.(2008)]{pad08} {Padmanabhan, N., Schlegel, D.~J., Finkbeiner, D.~P., Barentine, J.~C., Blanton, M.~R., Brewington, H.~J., Gunn, J.~E., Harvanek, M., Hogg, D.~W., Ivezi{\'c}, {\v Z}., Johnston, D., Kent, S.~M., Kleinman, S.~J., Knapp, G.~R., Krzesinski, J., Long, D., Neilsen, Jr., E.~H., Nitta, A., Loomis, C., Lupton, R.~H., Roweis, S., Snedden, S.~A., Strauss, M.~A., \& Tucker, D.~L. 2008, \apj, 674, 1217}
\end{thebibliography}
\clearpage



%% Use the figure environment and \plotone or \plottwo to include
%% figures and captions in your electronic submission.
%% To embed the sample graphics in
%% the file, uncomment the \plotone, \plottwo, and
%% \includegraphics commands
%%
%% If you need a layout that cannot be achieved with \plotone or
%% \plottwo, you can invoke the graphicx package directly with the
%% \includegraphics command or use \plotfiddle. For more information,
%% please see the tutorial on "Using Electronic Art with AASTeX" in the
%% documentation section at the AASTeX Web site, http://aastex.aas.org/
%%
%% The examples below also include sample markup for submission of
%% supplemental electronic materials. As always, be sure to check
%% the instructions to authors for the journal you are submitting to
%% for specific submissions guidelines as they vary from
%% journal to journal.

%% This example uses \plotone to include an EPS file scaled to
%% 80% of its natural size with \epsscale. Its caption
%% has been written to indicate that additional figure parts will be
%% available in the electronic journal.

%% Here we use \plottwo to present two versions of the same figure,
%% one in black and white for print the other in RGB color
%% for online presentation. Note that the caption indicates
%% that a color version of the figure will be available online.
%%


%% This figure uses \includegraphics to scale and rotate the still frame
%% for an mpeg animation.


%% If you are not including electonic art with your submission, you may
%% mark up your captions using the \figcaption command. See the
%% User Guide for details.
%%
%% No more than seven \figcaption commands are allowed per page,
%% so if you have more than seven captions, insert a \clearpage
%% after every seventh one.

%% Tables should be submitted one per page, so put a \clearpage before
%% each one.

%% Two options are available to the author for producing tables:  the
%% deluxetable environment provided by the AASTeX package or the LaTeX
%% table environment.  Use of deluxetable is preferred.
%%

%% Three table samples follow, two marked up in the deluxetable environment,
%% one marked up as a LaTeX table.

%% In this first example, note that the \tabletypesize{}
%% command has been used to reduce the font size of the table.
%% We also use the \rotate command to rotate the table to
%% landscape orientation since it is very wide even at the
%% reduced font size.
%%
%% Note also that the \label command needs to be placed
%% inside the \tablecaption.

%% This table also includes a table comment indicating that the full
%% version will be available in machine-readable format in the electronic
%% edition.

\clearpage

\begin{deluxetable}{ccrrrrrrrrcrl}
\tabletypesize{\scriptsize}
\rotate
\tablecaption{Sample table taken from \citet{treu03}\label{tbl-1}}
\tablewidth{0pt}
\tablehead{
\colhead{POS} & \colhead{chip} & \colhead{ID} & \colhead{X} & \colhead{Y} &
\colhead{RA} & \colhead{DEC} & \colhead{IAU$\pm$ $\delta$ IAU} &
\colhead{IAP1$\pm$ $\delta$ IAP1} & \colhead{IAP2 $\pm$ $\delta$ IAP2} &
\colhead{star} & \colhead{E} & \colhead{Comment}
}
\startdata
0 & 2 & 1 & 1370.99 & 57.35    &   6.651120 &  17.131149 & 21.344$\pm$0.006  & 2
4.385$\pm$0.016 & 23.528$\pm$0.013 & 0.0 & 9 & -    \\
0 & 2 & 2 & 1476.62 & 8.03     &   6.651480 &  17.129572 & 21.641$\pm$0.005  & 2
3.141$\pm$0.007 & 22.007$\pm$0.004 & 0.0 & 9 & -    \\
0 & 2 & 3 & 1079.62 & 28.92    &   6.652430 &  17.135000 & 23.953$\pm$0.030  & 2
4.890$\pm$0.023 & 24.240$\pm$0.023 & 0.0 & - & -    \\
0 & 2 & 4 & 114.58  & 21.22    &   6.655560 &  17.148020 & 23.801$\pm$0.025  & 2
5.039$\pm$0.026 & 24.112$\pm$0.021 & 0.0 & - & -    \\
0 & 2 & 5 & 46.78   & 19.46    &   6.655800 &  17.148932 & 23.012$\pm$0.012  & 2
3.924$\pm$0.012 & 23.282$\pm$0.011 & 0.0 & - & -    \\
0 & 2 & 6 & 1441.84 & 16.16    &   6.651480 &  17.130072 & 24.393$\pm$0.045  & 2
6.099$\pm$0.062 & 25.119$\pm$0.049 & 0.0 & - & -    \\
0 & 2 & 7 & 205.43  & 3.96     &   6.655520 &  17.146742 & 24.424$\pm$0.032  & 2
5.028$\pm$0.025 & 24.597$\pm$0.027 & 0.0 & - & -    \\
0 & 2 & 8 & 1321.63 & 9.76     &   6.651950 &  17.131672 & 22.189$\pm$0.011  & 2
4.743$\pm$0.021 & 23.298$\pm$0.011 & 0.0 & 4 & edge \\
\enddata
%% Text for table notes should follow after the \enddata but before
%% the \end{deluxetable}. Make sure there is at least one \tablenotemark
%% in the table for each \tablenotetext.
\tablecomments{Table \ref{tbl-1} is published in its entirety in the 
electronic edition of the {\it Astrophysical Journal}.  A portion is 
shown here for guidance regarding its form and content.}
\tablenotetext{a}{Sample footnote for table~\ref{tbl-1} that was generated
with the deluxetable environment}
\tablenotetext{b}{Another sample footnote for table~\ref{tbl-1}}
\end{deluxetable}

%% If you use the table environment, please indicate horizontal rules using
%% \tableline, not \hline.
%% Do not put multiple tabular environments within a single table.
%% The optional \label should appear inside the \caption command.

\clearpage


%% If the table is more than one page long, the width of the table can vary
%% from page to page when the default \tablewidth is used, as below.  The
%% individual table widths for each page will be written to the log file; a
%% maximum tablewidth for the table can be computed from these values.
%% The \tablewidth argument can then be reset and the file reprocessed, so
%% that the table is of uniform width throughout. Try getting the widths
%% from the log file and changing the \tablewidth parameter to see how
%% adjusting this value affects table formatting.

%% The \dataset{} macro has also been applied to a few of the objects to
%% show how many observations can be tagged in a table.

\clearpage

\begin{deluxetable}{lrrrrcrrrrr}
\tablewidth{0pt}
\tablecaption{Literature Data for Program Stars}
\tablehead{
\colhead{Star}           & \colhead{V}      &
\colhead{b$-$y}          & \colhead{m$_1$}  &
\colhead{c$_1$}          & \colhead{ref}    &
\colhead{T$_{\rm eff}$}  & \colhead{log g}  &
\colhead{v$_{\rm turb}$} & \colhead{[Fe/H]} &
\colhead{ref}}
\startdata
HD 97 & 9.7& 0.51& 0.15& 0.35& 2 & \nodata & \nodata & \nodata & $-1.50$ & 2 \\
& & & & & & 5015 & \nodata & \nodata & $-1.50$ & 10 \\
\dataset[ADS/Sa.HST#O6H04VAXQ]{HD 2665} & 7.7& 0.54& 0.09& 0.34& 2 & \nodata & \nodata & \nodata & $-2.30$ & 2 \\
& & & & & & 5000 & 2.50 & 2.4 & $-1.99$ & 5 \\
& & & & & & 5120 & 3.00 & 2.0 & $-1.69$ & 7 \\
& & & & & & 4980 & \nodata & \nodata & $-2.05$ & 10 \\
HD 4306 & 9.0& 0.52& 0.05& 0.35& 20, 2& \nodata & \nodata & \nodata & $-2.70$ & 2 \\
& & & & & & 5000 & 1.75 & 2.0 & $-2.70$ & 13 \\
& & & & & & 5000 & 1.50 & 1.8 & $-2.65$ & 14 \\
& & & & & & 4950 & 2.10 & 2.0 & $-2.92$ & 8 \\
& & & & & & 5000 & 2.25 & 2.0 & $-2.83$ & 18 \\
& & & & & & \nodata & \nodata & \nodata & $-2.80$ & 21 \\
& & & & & & 4930 & \nodata & \nodata & $-2.45$ & 10 \\
HD 5426 & 9.6& 0.50& 0.08& 0.34& 2 & \nodata & \nodata & \nodata & $-2.30$ & 2 \\
\dataset[ADS/Sa.HST#O5F654010]{HD 6755} & 7.7& 0.49& 0.12& 0.28& 20, 2& \nodata & \nodata & \nodata & $-1.70$ & 2 \\
& & & & & & 5200 & 2.50 & 2.4 & $-1.56$ & 5 \\
& & & & & & 5260 & 3.00 & 2.7 & $-1.67$ & 7 \\
& & & & & & \nodata & \nodata & \nodata & $-1.58$ & 21 \\
& & & & & & 5200 & \nodata & \nodata & $-1.80$ & 10 \\
& & & & & & 4600 & \nodata & \nodata & $-2.75$ & 10 \\
\dataset[ADS/Sa.HST#O56D06010]{HD 94028} & 8.2& 0.34& 0.08& 0.25& 20 & 5795 & 4.00 & \nodata & $-1.70$ & 22 \\
& & & & & & 5860 & \nodata & \nodata & $-1.70$ & 4 \\
& & & & & & 5910 & 3.80 & \nodata & $-1.76$ & 15 \\
& & & & & & 5800 & \nodata & \nodata & $-1.67$ & 17 \\
& & & & & & 5902 & \nodata & \nodata & $-1.50$ & 11 \\
& & & & & & 5900 & \nodata & \nodata & $-1.57$ & 3 \\
& & & & & & \nodata & \nodata & \nodata & $-1.32$ & 21 \\
HD 97916 & 9.2& 0.29& 0.10& 0.41& 20 & 6125 & 4.00 & \nodata & $-1.10$ & 22 \\
& & & & & & 6160 & \nodata & \nodata & $-1.39$ & 3 \\
& & & & & & 6240 & 3.70 & \nodata & $-1.28$ & 15 \\
& & & & & & 5950 & \nodata & \nodata & $-1.50$ & 17 \\
& & & & & & 6204 & \nodata & \nodata & $-1.36$ & 11 \\
\cutinhead{This is a cut-in head}
+26\arcdeg2606& 9.7&0.34&0.05&0.28&20,11& 5980 & \nodata & \nodata &$<-2.20$ & 19 \\
& & & & & & 5950 & \nodata & \nodata & $-2.89$ & 24 \\
+26\arcdeg3578& 9.4&0.31&0.05&0.37&20,11& 5830 & \nodata & \nodata & $-2.60$ & 4 \\
& & & & & & 5800 & \nodata & \nodata & $-2.62$ & 17 \\
& & & & & & 6177 & \nodata & \nodata & $-2.51$ & 11 \\
& & & & & & 6000 & 3.25 & \nodata & $-2.20$ & 22 \\
& & & & & & 6140 & 3.50 & \nodata & $-2.57$ & 15 \\
+30\arcdeg2611& 9.2&0.82&0.33&0.55& 2 & \nodata & \nodata & \nodata & $-1.70$ & 2 \\
& & & & & & 4400 & 1.80 & \nodata & $-1.70$ & 12 \\
& & & & & & 4400 & 0.90 & 1.7 & $-1.20$ & 14 \\
& & & & & & 4260 & \nodata & \nodata & $-1.55$ & 10 \\
+37\arcdeg1458& 8.9&0.44&0.07&0.22&20,11& 5296 & \nodata & \nodata & $-2.39$ & 11 \\
& & & & & & 5420 & \nodata & \nodata & $-2.43$ & 3 \\
+58\arcdeg1218&10.0&0.51&0.03&0.36& 2 & \nodata & \nodata & \nodata & $-2.80$ & 2 \\
& & & & & & 5000 & 1.10 & 2.2 & $-2.71$ & 14 \\
& & & & & & 5000 & 2.20 & 1.8 & $-2.46$ & 5 \\
& & & & & & 4980 & \nodata & \nodata & $-2.55$ & 10 \\
+72\arcdeg0094&10.2&0.31&0.09&0.26&12 & 6160 & \nodata & \nodata & $-1.80$ & 19 \\
\sidehead{I'm a side head:}
G5--36 & 10.8& 0.40& 0.07& 0.28& 20 & \nodata & \nodata & \nodata & $-1.19$ & 21 \\
G18--54 & 10.7& 0.37& 0.08& 0.28& 20 & \nodata & \nodata & \nodata & $-1.34$ & 21 \\
G20--08 & 9.9& 0.36& 0.05& 0.25& 20,11& 5849 & \nodata & \nodata & $-2.59$ & 11 \\
& & & & & & \nodata & \nodata & \nodata & $-2.03$ & 21 \\
G20--15 & 10.6& 0.45& 0.03& 0.27& 20,11& 5657 & \nodata & \nodata & $-2.00$ & 11 \\
& & & & & & 6020 & \nodata & \nodata & $-1.56$ & 3 \\
& & & & & & \nodata & \nodata & \nodata & $-1.58$ & 21 \\
G21--22 & 10.7& 0.38& 0.07& 0.27& 20,11& \nodata & \nodata & \nodata & $-1.23$ & 21 \\
G24--03 & 10.5& 0.36& 0.06& 0.27& 20,11& 5866 & \nodata & \nodata & $-1.78$ & 11 \\
& & & & & & \nodata & \nodata & \nodata & $-1.70$ & 21 \\
G30--52 & 8.6& 0.50& 0.25& 0.27& 11 & 4757 & \nodata & \nodata & $-2.12$ & 11 \\
& & & & & & 4880 & \nodata & \nodata & $-2.14$ & 3 \\
G33--09 & 10.6& 0.41& 0.10& 0.28& 20 & 5575 & \nodata & \nodata & $-1.48$ & 11 \\
G66--22 & 10.5& 0.46& 0.16& 0.28& 11 & 5060 & \nodata & \nodata & $-1.77$ & 3 \\
& & & & & & \nodata & \nodata & \nodata & $-1.04$ & 21 \\
G90--03 & 10.4& 0.37& 0.04& 0.29& 20 & \nodata & \nodata & \nodata & $-2.01$ & 21 \\
LP 608--62\tablenotemark{a} & 10.5& 0.30& 0.07& 0.35& 11 & 6250 & \nodata &
\nodata & $-2.70$ & 4 \\
\enddata
\tablenotetext{a}{Star LP 608--62 is also known as BD+1\arcdeg 2341p.  We will
make this footnote extra long so that it extends over two lines.}
%% You can append references to a table using the \tablerefs command.
\tablerefs{
(1) Barbuy, Spite, \& Spite 1985; (2) Bond 1980; (3) Carbon et al. 1987;
(4) Hobbs \& Duncan 1987; (5) Gilroy et al. 1988: (6) Gratton \& Ortolani 1986;
(7) Gratton \& Sneden 1987; (8) Gratton \& Sneden (1988); (9) Gratton \& Sneden 1991;
(10) Kraft et al. 1982; (11) LCL, or Laird, 1990; (12) Leep \& Wallerstein 1981;
(13) Luck \& Bond 1981; (14) Luck \& Bond 1985; (15) Magain 1987;
(16) Magain 1989; (17) Peterson 1981; (18) Peterson, Kurucz, \& Carney 1990;
(19) RMB; (20) Schuster \& Nissen 1988; (21) Schuster \& Nissen 1989b;
(22) Spite et al. 1984; (23) Spite \& Spite 1986; (24) Hobbs \& Thorburn 1991;
(25) Hobbs et al. 1991; (26) Olsen 1983.}
\end{deluxetable}

%% Tables may also be prepared as separate files. See the accompanying
%% sample file table.tex for an example of an external table file.
%% To include an external file in your main document, use the \input
%% command. Uncomment the line below to include table.tex in this
%% sample file. (Note that you will need to comment out the \documentclass,
%% \begin{document}, and \end{document} commands from table.tex if you want
%% to include it in this document.)

%% \input{table}

%% The following command ends your manuscript. LaTeX will ignore any text
%% that appears after it.

\end{document}

%%
%% End of file `sample.tex'.
